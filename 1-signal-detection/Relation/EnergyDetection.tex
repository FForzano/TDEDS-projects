% !TEX TS-program = pdflatex
% !TEX encoding = UTF-8 Unicode
\documentclass[%
    %corpo=11pt,  
    twoside, 
    a4paper
    ]{article}
\let\MakeUppercase\relax
\linespread{1.2}
\usepackage{geometry}
\geometry{a4paper, twoside, top=2.5cm, bottom=2.5cm, left=3cm, right=3cm, bindingoffset=0.5cm}

\usepackage[T1]{fontenc}
\usepackage[utf8]{inputenc}
\usepackage[main=english]{babel}
\usepackage{csquotes}

\usepackage{siunitx}
\usepackage{subcaption}
% Specific language
\usepackage{amsmath}
\usepackage{amsthm}
\usepackage{amssymb}
%\usepackage{ntheorem}
\usepackage{physics}
\usepackage{bm}

\usepackage{standalone}
\usepackage{pgfplots}
% and optionally (as of Pgfplots 1.3):
\pgfplotsset{compat=newest}
\pgfplotsset{plot coordinates/math parser=false}
\newlength\figureheight
\newlength\figurewidth

% Ref and bibliography
\usepackage{hyperref}

\DeclareMathOperator*{\argmax}{arg\,max}

\newcommand{\Set}[1]{\mathcal{#1}}
\newcommand{\Vector}[1]{\bm{\MakeLowercase{#1}}}
\newcommand{\Operator}[1]{\bm{\MakeUppercase{#1}}}
%%%%%%%%%%
\DeclareMathAlphabet{\mathsfbr}{OT1}{cmss}{m}{n}%for math sans serif (cmss)
\SetMathAlphabet{\mathsfbr}{bold}{OT1}{cmss}{bx}{n}%for math sans serif (cmss)
\DeclareRobustCommand{\msf}[1]{%
  \ifcat\noexpand#1\relax\msfgreek{#1}\else\mathsfbr{#1}\fi%for math sans serif (cmss)
}
\DeclareFontEncoding{LGR}{}{} % or load \usepackage{textgreek}
\DeclareSymbolFont{sfgreek}{LGR}{cmss}{m}{n}
\SetSymbolFont{sfgreek}{bold}{LGR}{cmss}{bx}{n}
\DeclareMathSymbol{\sXi}{\mathalpha}{sfgreek}{`X}
\DeclareMathSymbol{\sUpsilon}{\mathalpha}{sfgreek}{`U}
%%%%%%%%%%


\title{Experience 1: Energy detection}
\author{Federico Forzano}
\date{}

\begin{document}
\maketitle

\section{Introduction}
  This experience is based on the energy detection process proposed in the paper \cite{BarDaiConWin:J15} for
  ranging applications. 
  The goal of the energy detection is to find the presence or the absence of the signal in an energy bin. The
  detection of the signal in a sequence of bins allows us to develop algorithms for the detection of the main path
  in wireless propagation channels that allow us to extimate, for example, the time of fly of the signal and 
  then the distance of the signal source.

  The experience has mainly two goals:
  \begin{itemize}
    \item verify with a simulation that the statistical pdf of the $i-th$ bin random variable 
    $\msf{B}_i$ is a $\chi^2$ pdf;
    \item evaluate the performance of two different algorithms for the detection of the signal in a sequence
          of bins: the \emph{maximum bin search} (MBS) and the \emph{threshold crossing search} (TCS).
  \end{itemize}
  
\section{Elements of ranging systems}
  In order to estimate the distance of a signal source from the received signal, one of the possible ways is to evaluate
  the energy level of the last one. 
  In particular we suppose that the transmitter emits $N_p$ copies of a signal $s(t)$. The aim of a ranging 
  system is to estimate the delay time $\tau$ before the signal is received by the receiver which performs 
  $N_p$ obseravations each with duration $T_{obs}$. Given the time $\tau$, it is immediate to find the estimation
  of the distance $d = c \tau$, where c is, usually, the speed of light in vacuum constant.
  In this process the key role is to correctly find the delay time $\tau$ of the arrival of the main signal,
  in the general case with possible reflection or rifraction of the signal. The choosen approach is to evaluate
  the bin of energy of the received signal.
  
  If we suppose that the received signal can be written as
  \begin{equation}
    r(t) = u(t) + n(t),
  \end{equation}
  where $u(t)$ is the received signal (which is not equal to $s(t)$ in general) and $n(t)$ is AWGN with 
  variance $\sigma^2$ and average $0$; we can define
  the $i-th$ energy bin $\beta_{i,p}$ for the $p-th$ observation as:
  \begin{equation}
    \beta_{i,p} = \sum_{s=0}^{N_{sb}-1} r^2(t_{i,p,s}),
  \end{equation}
  where $t_{i,p,s}$ is the instant of the $(i+s)-th$ sample of the signal for the $p-th$ observation and 
  $N_{sb}$ is a project parameter. The total number of bins results $N_b$. 
  The bins obtained for each observation are then processed by an averaging block and we obtain a bins array
  $\Vector{b} = \left[b_0, b_1, \dots, b_{N_b-1}\right]$.
  
  The $i-th$ bin $b_i$ is an instantiation of the RV $\msf{B}_i$ and it can be proven (we will verify
  empirically in the next section), that
  \begin{equation}
    \msf{B}_i \sim \frac{\sigma^2}{N_p} \chi^2_{N_p N_{sb}}(\lambda_i)
    \label{eq:bin_dist}
  \end{equation}
  where $N_p N_{sb}$ is the freedom degrees and $\lambda_i$ is the non-centrality parameter of the $\chi^2$
  defined as
  \begin{equation}
    \lambda_i = \sum_{p=0}^{N_p-1} \sum_{s=0}^{N_{sb}-1} \frac{u_{i,p,s}^2}{\sigma^2}.
    \label{eq:lambda_par}
  \end{equation}

  For the decision about which bin is the one which contains the main signal we can use different algorithms.
  The two which we will test are the \emph{threshold crossing search} (TCS) and the \emph{maximum bin search}
  (MBS).
  Both algorithms analize only the bins which exceeded a given threshold $\xi_i$ (the footer $i$ is because it can 
  be different for each bin, in general), if at least one does it (event $\mathcal{C}_{th}$). The difference
  between them is that the TCS algorithm choose the $\hat{\imath}-th$ bin if:
  \begin{equation}
    \hat{\imath} = \min\{i \in \mathcal{B} | b_i > \xi_i\},
  \end{equation}
  where $\mathcal{B}$ is the set of all possible $i$;
  the MBS indeed choose the higher bin $b_{\hat{\imath}}$ which exceeded the threshold, i.e:
  \begin{equation}
    \hat{\imath} = \argmax_{i \in \mathcal{B}}\{b_i\}.
  \end{equation}

  %
  \begin{figure}[t]
    \begin{subfigure}[t]{0.49\linewidth}
      \includestandalone{./Images/signal_bin}
      \caption{Comparison for the bin with signal. The choosen SNR is $\gamma = 15 \si{dB}$.}
      \label{fig:signal_bin}
    \end{subfigure}
    \begin{subfigure}[t]{0.49\linewidth}
      \includestandalone{./Images/noise_bin}
      \caption{Comparison for the only noise bin.}
      \label{fig:noise_bin}
    \end{subfigure}
    \caption{Comparison between theoretical bin pdf $f_{\msf{B}}(b)$ and empirical pdf for an only noise 
    bin and for a bin with signal. The continuous lines represent the theoretical distributions, the histograms
    plot the number of occurences of each instantiations value, normalized over $N$. The choosen number 
    of instantiations of each RV of the signal sample is $N=10000$ and the number of samples for bin is 
    $N_{sb} = 5$.}
  \end{figure}
  %
\section{Results}
  We introduce now the results of the experience. The first goal is to test that the empirical distribution,
  obtained with a simulation, of a bin tends to the distribution \ref{eq:bin_dist} for a large number of 
  instantiations $N$. The second goal is to check the performances of the two algorithms TSC and MBS, for different
  levels of the threshold noise ratio $\rho$ ($\rho = \frac{\xi^2}{\sigma^2}$), by varying the signal noise ratio
  $\gamma$.
  
  
  \subsection{Bin distribution}
    
    In order to check the distribution given in \ref{eq:bin_dist}, we have simulated $N=10000$ instantiations of a signal sample
    with and without signal energy (only noise bin). We assume a SNR $\gamma = 15 \si{dB}$ for the bin with signal 
    and we choose $N_{sb}=5$.
    It is supposed that the noise has a variance $\sigma^2 = 1$ and that only one transmission is performed
    ($N_p = 1$).
    The parameter \ref{eq:lambda_par} becomes, for the bin with signal
    \begin{equation}
      \lambda_i = A^2,
      \label{eq:lambda_signal}
    \end{equation}
    where $A$ is the amplitude of the signal obtained as $A = \sqrt{\gamma \sigma^2}$, and 
    \begin{equation}
      \lambda_i = 0
      \label{eq:lambda_noise}
    \end{equation}
    for the only noise bin.

    As we can see in the figure \ref{fig:signal_bin}, for the bin with the signal energy, and in the figure
    \ref{fig:noise_bin}, for the only noise bin; the histograms which plot the occurences for 
    each value of the instantiations (normalized over $N$), obtained with a simulation,
    follow the theoretical functions (continuous lines), obtained with the equation 
    \ref{eq:bin_dist} with the $\lambda$ values calculated as in the equations \ref{eq:lambda_signal} and
    \ref{eq:lambda_noise}.

  %
  \begin{figure}[t]
    \centering
    \begin{subfigure}[t]{0.49\linewidth}
      \includestandalone{./Images/threshold_crossing_search}
      \caption{TCS algorithm simulation.}
      \label{fig:TCS}
    \end{subfigure}
    \begin{subfigure}[t]{0.49\linewidth}
      \includestandalone{./Images/maximum_bin_search}
      \caption{MBS algorithm simulation.}
      \label{fig:MBS}
    \end{subfigure}
    \caption{Estimated probability of success $\hat{P}_s$ in the bin with signal research as function of $\gamma$ for 
    different values of $\rho$. The number 
    of instantiations of each RV of the signal sample choosen is $N=10000$ and the number of samples for bin is 
    $N_{sb} = 5$.}
  \end{figure}
  %
  %
  \begin{figure}[t]
    \centering
    \includestandalone{./Images/threshold_plot}
    \caption{Thresholds position over the theoretical noise and signal bin distributions. Bin distributions are 
    obtained with $\sigma^2 = 1$ and $\gamma = 15\si{dB}$ for the signal bin.
    The dashed lines show the threshold levels over the distributions (filled plots).}
    \label{fig:TCS_explain1}

  \end{figure}
  %
  \subsection{Algorithms performance}
    The performances evaluation of the two algorithms (TCS and MBS) were carried out with a software simulation 
    that realizes $N = 10000$ instantiations of the $N_s = 100$ samples of the received signal, for each 
    $\gamma \in \left[ 0\si{dB}, 25\si{dB} \right]$ and for each $\rho \in \{ 10\si{dB}, 20\si{dB}, 30\si{dB}, 
    40\si{dB} \}$. We assume that 
    only one sample contains the transmitted signal i.e, there are not reflections in the communication channel.
    For each instantiation set, from the signal samples we have computed the bin samples and then the two 
    algorithms have been applied. We have counted the number of successes in the bin research for $N$ instantiations 
    and then we have estimated the probability of success $P_s$ as the success rate:
    \begin{equation}
      \hat{P}_s = \frac{\# successes}{N}
    \end{equation}
    The success event is verified when the recognized $\hat{\imath}$ is the same as the bin with signal $i^*$.
    We have assumed that the noise is AWGN with variance $\sigma^2 = 1$ and we have choosen $N_{sb} = 5$ and 
    $N_p = 1$.

    In figure \ref{fig:TCS} we can see the plot of the performance for the TCS algorithm where, on the \mbox{x-axis}
    there are the SNR $\gamma$ in $\si{dB}$ scale and in the \mbox{y-axis} there is the estimated success probability
    $\hat{P}_s$. We can observe that for $\rho = 10 \si{dB}$ the $\hat{P}_s$ is near to $0$ for all $\gamma$.
    If we look to the figure \ref{fig:TCS_explain1}, that plots the thresholds over the pdfs of noise and 
    signal bins for $\gamma = 15 \si{dB}$, it is clear that both noise and signal can be over this threshold. 
    The decisor so, can easily fail in its job. 
    For the others $\rho$ we can see that when $\gamma$ is less then $\rho$ nothing can be recognize and so 
    $\hat{P}_s \simeq 0$. When instead $\gamma$ is greater the $\rho$ the success probability $\hat{P}_s$ is 
    as near to $1$ as threshold is high but higher threshold implies that we need more SNR for a correct operation
    of the decisor.
    In figure \ref{fig:TCS_explain1} we can see that:
    \begin{itemize}
      \item with $\gamma = 20 \si{dB}$, a small portion of noise is over 
      $\rho = 20 \si{dB}$ (i.e there is a little probability that noise exceed the threshold) and the signal is 
      all over the threshold. The $\hat{P}_s$ is so lower then $1$ because noise can be recognized as signal.
      \item With $\gamma = 30 \si{dB}$ all the noise is under the threshold and only a portion of signal is over that.
      The probability of success $\hat{P}_s$ is so lower than $1$ because the signal can be unrecognized.
      \item With $\gamma = 40 \si{dB}$ neither the noise and the signal are over the threshold. The $\hat{P}_s$ so is 
      equal (or near) to $0$.
    \end{itemize}

    In figure \ref{fig:MBS} we can see the performance of the MBS algorithm. We can notice that, indeed the TCS
    algorithm, for a $\gamma$ high enough, the signal is always correctly recognized and so $\hat{P}_s \simeq 1$. When 
    $\gamma$ is lower instead, the signal can be confused with the noise and the $\hat{P}_s$ is lower than $1$.
    For the highest $\rho$, the $\hat{P}_s$ for low $\gamma$ is near to $0$ because the signal is all under the 
    threshold (as for the TCS decisor).
     

\newpage
\bibliographystyle{IEEEtran}
\bibliography{bibliography}
\end{document}