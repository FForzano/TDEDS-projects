\documentclass{standalone}
% This file was created with tikzplotlib v0.9.15.

\usepackage{siunitx}
\usepackage{pgfplots}
% and optionally (as of Pgfplots 1.3):
\pgfplotsset{compat=newest}
\pgfplotsset{plot coordinates/math parser=false}
\newlength\figureheight
\newlength\figurewidth

\newcommand{\Set}[1]{\mathcal{#1}}
\newcommand{\Vector}[1]{\bm{\MakeLowercase{#1}}}
\newcommand{\Operator}[1]{\bm{\MakeUppercase{#1}}}
%%%%%%%%%%
\DeclareMathAlphabet{\mathsfbr}{OT1}{cmss}{m}{n}%for math sans serif (cmss)
\SetMathAlphabet{\mathsfbr}{bold}{OT1}{cmss}{bx}{n}%for math sans serif (cmss)
\DeclareRobustCommand{\msf}[1]{%
  \ifcat\noexpand#1\relax\msfgreek{#1}\else\mathsfbr{#1}\fi%for math sans serif (cmss)
}
\DeclareFontEncoding{LGR}{}{} % or load \usepackage{textgreek}
\DeclareSymbolFont{sfgreek}{LGR}{cmss}{m}{n}
\SetSymbolFont{sfgreek}{bold}{LGR}{cmss}{bx}{n}
\DeclareMathSymbol{\sXi}{\mathalpha}{sfgreek}{`X}
\DeclareMathSymbol{\sUpsilon}{\mathalpha}{sfgreek}{`U}


% Style to select only points from #1 to #2 (inclusive)
\pgfplotsset{select coords between index/.style 2 args={
    x filter/.code={
        \ifnum\coordindex<#1\def\pgfmathresult{}\fi
        \ifnum\coordindex>#2\def\pgfmathresult{}\fi
    }
}}


\begin{document}

% This file was created with tikzplotlib v0.9.15.
\begin{tikzpicture}

\definecolor{color0}{rgb}{0.12156862745098,0.466666666666667,0.705882352941177}

\begin{axis}[
tick align=outside,
tick pos=left,
x grid style={white!69.0196078431373!black},
xlabel={number of features},
xmin=0, xmax=6,
xtick style={color=black},
y grid style={white!69.0196078431373!black},
ylabel={percentage of variance covered},
ymin=0, ymax=100,
ytick style={color=black}
]
\addplot [dashed, semithick, color0, mark=o, mark options={solid,color0}, select coords between index={0}{15}]
table {Eigenvalues_dimension_criterium_1.txt};
\end{axis}

\end{tikzpicture}
\end{document}