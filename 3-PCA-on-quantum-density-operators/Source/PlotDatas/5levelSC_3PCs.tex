\documentclass{standalone}
% This file was created with tikzplotlib v0.9.15.

\usepackage{siunitx}
\usepackage{pgfplots}
% and optionally (as of Pgfplots 1.3):
\pgfplotsset{compat=newest}
\pgfplotsset{plot coordinates/math parser=false}
\newlength\figureheight
\newlength\figurewidth

\newcommand{\Set}[1]{\mathcal{#1}}
\newcommand{\Vector}[1]{\bm{\MakeLowercase{#1}}}
\newcommand{\Operator}[1]{\bm{\MakeUppercase{#1}}}
%%%%%%%%%%
\DeclareMathAlphabet{\mathsfbr}{OT1}{cmss}{m}{n}%for math sans serif (cmss)
\SetMathAlphabet{\mathsfbr}{bold}{OT1}{cmss}{bx}{n}%for math sans serif (cmss)
\DeclareRobustCommand{\msf}[1]{%
  \ifcat\noexpand#1\relax\msfgreek{#1}\else\mathsfbr{#1}\fi%for math sans serif (cmss)
}
\DeclareFontEncoding{LGR}{}{} % or load \usepackage{textgreek}
\DeclareSymbolFont{sfgreek}{LGR}{cmss}{m}{n}
\SetSymbolFont{sfgreek}{bold}{LGR}{cmss}{bx}{n}
\DeclareMathSymbol{\sXi}{\mathalpha}{sfgreek}{`X}
\DeclareMathSymbol{\sUpsilon}{\mathalpha}{sfgreek}{`U}


% Style to select only points from #1 to #2 (inclusive)
\pgfplotsset{select coords between index/.style 2 args={
    x filter/.code={
        \ifnum\coordindex<#1\def\pgfmathresult{}\fi
        \ifnum\coordindex>#2\def\pgfmathresult{}\fi
    }
}}


\begin{document}

% This file was created with tikzplotlib v0.9.15.
\begin{tikzpicture}

\definecolor{color0}{rgb}{0.12156862745098,0.466666666666667,0.705882352941177}
\definecolor{color1}{rgb}{1,0.498039215686275,0.0549019607843137}
\definecolor{color2}{rgb}{0.172549019607843,0.627450980392157,0.172549019607843}
\definecolor{color3}{rgb}{0.83921568627451,0.152941176470588,0.156862745098039}
\definecolor{color4}{rgb}{1,0,1}

\begin{axis}[
width=1\linewidth,
legend cell align={left},
legend style={
    fill opacity=0.8,
    draw opacity=1,
    text opacity=1,
    at={(0.67,0.99)},
    anchor=north west,
    draw=white!80!black
},
tick align=outside,
tick pos=left,
x grid style={white!69.0196078431373!black},
% xmin=0, xmax=1,
xtick style={color=black},
xlabel={Principal component 1},
y grid style={white!69.0196078431373!black},
% ymin=-1, ymax=1,
ytick style={color=black},
ylabel={Principal component 2},
zlabel={Principal component 3},
view={65}{30},
grid
]
\addplot3 [only marks, color=color0, select coords between index={0}{19}] table  {./5levelSC_3PCs.txt};
\addlegendentry{$\mu=1.5$, $\zeta=0.2 e^{i\pi}$}
\addplot3 [only marks, color=color1, select coords between index={20}{39}] table  {./5levelSC_3PCs.txt};
\addlegendentry{$\mu=1.5$, $\zeta=0.5 e^{i\pi}$}
\addplot3 [only marks, color=color2, select coords between index={40}{59}] table  {./5levelSC_3PCs.txt};
\addlegendentry{$\mu=2.0$, $\zeta=0.2 e^{i\pi}$}
\addplot3 [only marks, color=color3, select coords between index={60}{79}] table  {./5levelSC_3PCs.txt};
\addlegendentry{$\mu=2.0$, $\zeta=0.5 e^{i\pi}$}
\addplot3 [only marks, color=color4, select coords between index={80}{99}] table  {./5levelSC_3PCs.txt};
\addlegendentry{$\mu=0.0$, $\zeta=0$}
\end{axis}

\end{tikzpicture}
\end{document}